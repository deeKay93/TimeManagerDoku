% Colors
\definecolor{ListingBackground}{HTML}{E6E6E6}
\definecolor{LinkColor}{HTML}{000000}


% Font
\usepackage[onehalfspacing]{setspace}
\usepackage{lmodern}
\usepackage[official]{eurosym}
\usepackage{enumitem}
\usepackage[locale=DE]{siunitx} % SI Units für Währungen
\DeclareSIUnit{\EUR}{\text{\euro}} % Beispielverwendung: \SI{10.10}{\EUR}

\usepackage[autostyle=true,german=quotes]{csquotes}
\usepackage{url}
\newcommand{\code}[1]{\texttt{#1}}

% Keine Einrückungen am Zeilenanfang
\setlength\parindent{0pt}

% Additional Setup
\usepackage[unicode=true,hypertexnames=false,colorlinks=true,linkcolor=LinkColor,citecolor=LinkColor,urlcolor=LinkColor,pdftex]{hyperref}

% Trennung von URLs im Literaturverzeichnis (große Werte [> 10000] verhindern die Trennung)
\defcounter{biburlnumpenalty}{10} % Strafe für Trennung in URL nach Zahl
\defcounter{biburlucpenalty}{500}  % Strafe für Trennung in URL nach Großbuchstaben
\defcounter{biburllcpenalty}{500}  % Strafe für Trennung in URL nach Kleinbuchstaben
\interfootnotelinepenalty=10000 % prevent all footnotes from breaking over a page.

% Configs
\setcounter{tocdepth}{1} % Limit table of contents to subsection
\sisetup{detect-weight=true, detect-family=true} % SI Units shall detect font weight and family
\setlist[description]{style=nextline} % Break definitions of terms to a new line (used by \begin{description} \item[foo] bar \end{description})
\renewcommand*{\bibfont}{\small}

%Hurenkinder und Schusterjungen vermeiden
\clubpenalty = 10000
\widowpenalty = 10000
\displaywidowpenalty = 10000

% Quellcode
\usepackage{listings}
\usepackage{float}
\usepackage{textcomp}
\lstset{
    inputpath=assets/listings,
    language=Java,			% Standardsprache des Quellcodes
    numbers=left,			% Zeilennummern links
    stepnumber=1,			% Jede Zeile nummerieren.
    numbersep=5pt,			% 5pt Abstand zum Quellcode
    numberstyle=\tiny,		% Zeichengrösse 'tiny' für die Nummern.
    breaklines=true,		% Zeilen umbrechen wenn notwendig.
    breakautoindent=true,	% Nach dem Zeilenumbruch Zeile einrücken.
    postbreak=\space,		% Bei Leerzeichen umbrechen.
    tabsize=2,				% Tabulatorgrösse 2
    basicstyle=\ttfamily\footnotesize, % Nichtproportionale Schrift, klein für den Quellcode
    showspaces=false,		% Leerzeichen nicht anzeigen.
    showstringspaces=false,	% Leerzeichen auch in Strings ('') nicht anzeigen.
    extendedchars=true,		% Alle Zeichen vom Latin1 Zeichensatz anzeigen.
    captionpos=b,			% sets the caption-position to bottom
    backgroundcolor=\color{ListingBackground}, % Hintergrundfarbe des Quellcodes setzen.
    xleftmargin=0pt,		% Rand links
    xrightmargin=0pt,		% Rand rechts
    frame=single,			% Rahmen an
    frameround=ffff,
    rulecolor=\color{darkgray},	% Rahmenfarbe
    fillcolor=\color{ListingBackground},
    keywordstyle=\color[rgb]{0.133,0.133,0.6},
    commentstyle=\color[rgb]{0.133,0.545,0.133},
    stringstyle=\color[rgb]{0.627,0.126,0.941},
    float,
    upquote=true
}


\colorlet{punct}{red!60!black}
\definecolor{delim}{RGB}{20,105,176}
\colorlet{numb}{magenta!60!black}

\lstdefinelanguage{TypeScript}{
    keywords={typeof, new, true, false, catch, function, return, null, catch,
    switch, var, if, in, while, do, else, case, break, number, string, Date},
    keywordstyle=\color{delim}\bfseries,
    ndkeywords={class, export, boolean, throw, implements, import, this, type, defaults, extends},
    ndkeywordstyle=\color{magenta}\bfseries,
    identifierstyle=\color{black},
    sensitive=true,
    morecomment=[s]{<}{>}, % Actually not a Comment but a Tag
    commentstyle=\color{red}\ttfamily,
    stringstyle=\color{red}\ttfamily,
    morestring=[b]',
    morestring=[b]"
}


\lstdefinelanguage{json}{
    string=[s]{"}{"},
    stringstyle=\color{numb},
    literate=
     *{0}{{{\color{numb}0}}}{1}
      {1}{{{\color{numb}1}}}{1}
      {2}{{{\color{numb}2}}}{1}
      {3}{{{\color{numb}3}}}{1}
      {4}{{{\color{numb}4}}}{1}
      {5}{{{\color{numb}5}}}{1}
      {6}{{{\color{numb}6}}}{1}
      {7}{{{\color{numb}7}}}{1}
      {8}{{{\color{numb}8}}}{1}
      {9}{{{\color{numb}9}}}{1}
      {:}{{{\color{punct}{:}}}}{1}
      {,}{{{\color{punct}{,}}}}{1}
      {\{}{{{\color{delim}{\{}}}}{1}
      {\}}{{{\color{delim}{\}}}}}{1}
      {[}{{{\color{delim}{[}}}}{1}
      {]}{{{\color{delim}{]}}}}{1}
}

\lstdefinelanguage{yaml}{
  identifierstyle=\color{delim},
  sensitive=false,
  comment=[l]{\#},
  commentstyle=\color{purple}\ttfamily,
  stringstyle=\color{numb},
  morestring=[b]',
  morestring=[b]"
}

\lstdefinelanguage{firestoreRule}{
  identifierstyle=\color{delim},
  keywords={},
  otherkeywords={% Operators
    =, ==, /
  },
   % list of keywords
   keywords= [2]{
    service,
    match,
    allow,
    for
  },
  keywordstyle=\color{purple},
  keywordstyle=[2]\color{numb},
  sensitive=false
}

\lstloadlanguages{Python,Java,bash}



% Algorithmen (Pseudocode)
\usepackage[german,onelanguage,linesnumbered,ruled]{algorithm2e}
% Styling Kommentar
\newcommand\mycommfont[1]{\color{gray}\ttfamily{#1}}
\SetCommentSty{mycommfont}

% Runde Klammern um Bedingungen
\SetKwIF{If}{ElseIf}{Else}{wenn~(}{)~dann}{sonst wenn~(}{sonst}{}
\SetKwFor{ForEach}{für jedes~(}{)~tue}{Ende}

%Eingabe und Ausgabe Texte
\SetKwInOut{Input}{Eingabe}
\SetKwInOut{Output}{Ausgabe}
\section{Besonderheiten (Daniel)}
\subsection{Firebase}
In dieser Applikation wird Firebase für die Authentifizierung,
die Datenspeicherung, sowie Benachrichtigungen genutzt.
Die ersten beiden Aspekte werden im folgenden vorgestellt.
Die Benachrichtigungen weren in \autoref{sec:besonderheiten-benachrichtigung} genauer betrachtet.

\paragraph{Authentifizierung}
Die Authentifizierung erfolgt über die von Firebase (bzw. React Native Firebase) zur Verfügung gestellten Schnittstellen.
Jeder Nutzer erhält bei Anmeldung eine UID, über welche er eindeutig identifiziert werden kann.

Firebase unterstützt zur Anmeldung Anonym, E-Mail/Passwort, Telefon, Google, Play Spiele, Facebook, Twitter und GitHub.
In dieser Anwendung wird jedoch nur von den ersten beiden Gebrauch gemacht.



\paragraph{Datenspeicherung}
Wie bereits beschrieben werden die Daten im \textit{Cloud Firestore} gespeichert.
Auf oberster Ebene werden Sammlungen angelegt.
Eine solche Sammlung enthält beliebig viele Dokumente.
Ein Dokument wiederum kann Daten in einem JSON-ähnlichen Format und weitere Sammlungen enthalten.
So lässt sich die gesamte Daten aus wechselnden Sammlungen und Dokumenten strukturieren.

Die Struktur für diesen Programmentwurf ist in \autoref{app-datenbank} dargestellt und wird im folgenden beschrieben:

Auf oberster Ebene werden unter der Sammlung \texttt{users} alle Benutzer eingeordnet.
Für jeden der Benutzer wird ein Dokument mit dessen UID angelegt.
Diese Dokument besitzt ein Feld für den Namen des Nutzers,
sowie die Sammlungen \texttt{categories} und \texttt{holidays}.

\texttt{categories} enthält für jede Kategorie ein Dokument mit automatisch generierter ID angelegt.
In diesem Dokument befinden sich neben dem Namen alle für die Zeiterfassung relevanten Daten.
Auf diese wird im Verlauf dieser Arbeit genauer eingegangen.
Außerdem enthält eine Kategorie-Dokument die Sammlung \texttt{times}.
Diese wiederum enthält für jede getätigte Zeiterfassung ein Dokument mit Start- und Stoppzeit sowie den gesamten Minuten.

Die Sammlung \texttt{holidays} enthält alle vom Nutzer gepflegten freie Tage mit den notwendigen Informationen.

Um sicherzustellen, dass nur angemeldete Nutzer Zugriff auf ihre Daten haben,
können sogenannte \textit{Regeln} angelegt werden.
Die Regel für diesen Programmentwurf ist in \autoref{lst:firestore-rule} dargestellt.
Über die ersten beiden Zeilen wird die Datenbank an sich addressiert.
In durch die Zeilen 3 und 4 wird geprüft, ob die Nutzer-UID mit dem Titel des Dokuments übereinstimmt.
ist dies der Fall, so werden Lese- und Schreibrechte gestattet.
In Zeile 5 wird durch \lstinline[language=firestoreRule]{**} angegeben,
dass die darauffolgende Regel für alle darunterliegenden Dokumente gilt.
In diesem Fall wird die Regel aus Zeile 4 wiederholt.

\lstinputlisting[
    caption=Regeln für Firestore,
    label=lst:firestore-rule,
    language=firestoreRule,
    float=ht
]{firestoreRule.txt}


Insgeamt besagt die Regel,
dass ein angemeldeter Benutzer auf das zu seinem Benutzer zugehörige Dokument,
sowie alle darunterliegenden Dokumente volle Schreib- und Leseberechtigungen erhält.
Für eine produktive Anwendung könnten diese Regeln noch ergänzt werden.
Beispielsweise sind strengere Regeln zur Einhaltung der Datanbankstruktur oder Validierungen der eingegebenen Daten denkbar.
Für die Entwicklung eines Prototyps sind solch strenge Regeln allerdings eher hinderlich.


\subsection{Login}
Die Authentifizierung an sich lässt sich dank Firebase unkompliziert umsetzen.
Auch ein Formular, welches einem Nutzer gestattet sich anzumelden oder zu registieren ist einfach zu realisieren.

Es bleiben somit noch zwei Aufgaben, welche es zu lösen gilt:
Zum einen sollte ein bereits angemeldeter Benutzer sich an seinem Gerät nicht immer wieder neu anmelden müssen.
Zum anderen sollte der Anwender erst nach Anmeldung Zugriff auf die inneren Funktionen erhalten.

Ersteres ist standardmäßig in React Native Firebase vorgesehen.
Ein Nutzer bleibt auch nach Beenden der App angemeldet.
Beim Erneuten start muss dann überprüft werden, ob bereits ein Nutzer angemeldet ist.
Dies geschieht über die Funktion \texttt{checkLoggedIn} (\autoref{lst:checkLoggedIn}),
welche beim Start der Anwendung aufgerufen wird.
Ist ein Nutzer angemeldet, so wird die Redux-Action \emph{userSignInSuccess},
ansonten \emph{userSignOutSuccess} gefeuert.
Die Applikation kann dann entsprechend reagieren.

\lstinputlisting[
    caption=checkLoggedIn,
    label=lst:checkLoggedIn,
    language=TypeScript,
    float=ht
]{checkLoggedIn.ts}

Die Komponente \texttt{LoginDeciderComponent} (siehe \autoref{app-LoginDecider}) sorgt dafür,
dass nur angemeldete Nutzer auf die eigentliche Applikation zugreifen können.
Diese Komponente erhält den aktuellen Login-Status aus dem Redux-State und zeigt eine bestimmte Unterkomponente an.
Die Zordnung zwischen Satus und Unterkomponente ist in \autoref{table:login-states} aufgelistet.
Ein \text{Splash Screen} ist ein einfacher Bildschirm, welcher lediglich das Logo der Anwendung enthält.

\begin{table}[h!]
    \centering
     \begin{tabular}{| l | c | c |}
        \hline
        Status      & Bedeutung                      & Unterkomponente \\
        \hline\hline
        checking    & Prüfung ob Nutzer angemeldet   & Splash Screen\\
        \hline
        logged in   & Nutzer ist angemeldet          & \multirow{2}{*}{Eigentliche Applikation}\\
        \cline{1-2}
        logging out & Nutzer wird abgemeldet         & \\
        \hline
        logged out  & Nutzer ist abgemeldet          & \multirow{2}{*}{Login Screen} \\
        \cline{1-2}
        logging in  & Nutzer wird angemeldet         & \\
        \hline
     \end{tabular}
     \caption{Login Stati}
     \label{table:login-states}
\end{table}

Insgesamt wird durch Kombination der beiden Lösungen sichergestellt,
dass ein bereits angemeldeter Nutzer sich nicht nochmals anmelden muss (bzw. kann)
und dass eine nicht angemeldeter Nutzer lediglich auf den Login zugreifen kann.

\subsection{Freie Tage (Lea)}

\subsection{Zeiterfassung}
Die Zeiterfassung ist die zentrale Funktion der Anwendung.
Hauptsächlich wird diese über das Starten und Stoppen der Zeitmessung einer bestimmten Kategorie durchgeführt.
Es ist aber auch möglich Zeiten manuell zu erfassen.
Außerdem ist es notwendig die Gesamtzeit Applikationsseitig anzupassen.
So muss bei Intervall-Kategorien die Gesamtzeit regelmäßig zurückgesetzt werden
und bei Endlos-Kategorien die Gesamtzeit um die entsprechenden Sollstunden reduziert werden.

\paragraph{Starten der Zeitmessung}
Beim Starten der Zeitmessung einer Kategorie wird diese Tatsache,
sowie der Zeitpunkt gespeichert.
Zusätzlich wird das Senden von Benachrichtigungen vorbereitet (siehe \autoref{sec:besonderheiten-benachrichtigung}).
Durch das Speichern wird sichergestellt, dass die Zeitmessung geräteunabhängig abläuft.

\paragraph{Stoppen der Zeitmessung}
Stoppt der Nutzer die Zeitmessung,
so wird aus Start- und Stoppzeit der Messung die vergangene Zeit berechnet.
Diese wird zur Gesamtzeit hinzuaddiert.

\paragraph{Manuelle Erfassung}
Bei der manuellen Zeiterfassung muss zwischen Intervall- und Endlos-Kategorien unterschieden werden.
Während bei Endlos-Kategorien die Gesamtzeit einfach weiter erhöht werden kann,
muss bei Intervallen zunächst geprüft werden,
ob sich die manuell erfasste Zeit innerhalb des aktuellen Intervalls befindet.
Ansonsten wird die Erfassung zwar hinzugefügt,
die Gesamtzeit ändert sich allerdings nicht.

\paragraph{Anpassung der Gesamtzeit}
Zur Anpassung der Gesamtzeit erhält jede Kategorie ein Feld \emph{lastUpdated}.
Dieses enthält den Tag, an welchem die Gesamtzeit zuletzt durch die Applikation angepasst wurde.
Bei Intervall-Kategorien entspricht dies dem Tag, an welchem sie zuletzt zurückgesetzt wurde.
Bei endlosen Kategorien dem letzen Tag, an welchem das letze Mal die Anapassung stattfand.

Jedes Mal, wenn die Daten der Kategorien aus der Datenbank geladen werden,
wird für jede Kategorie eine Überprüfung und Anapassung durchgeführt.
Für endlose Kategorien geschieht dies nach \autoref{fig:anpassung-endlos},
für Intervall-Kategorien nach \autoref{fig:anpassung-intervall}.
\begin{figure}[ht!]
    \centering
    \resizebox{0.5\textwidth}{!}{
        \input{assets/img/diagramme/ZeitAnspassungIntervall.latex}
    }
	\caption{Anapssung der Gesamtzeit einer Intervall-Kategorie}
    \label{fig:anpassung-intervall}
\end{figure}

\begin{figure}
    \begin{subfigure}{0.5\textwidth}
        \resizebox{0.9\linewidth}{!}{
            \input{assets/img/diagramme/ZeitAnspassungEndless.latex}
        }
		\caption{Rahmenablauf}
	\end{subfigure}
	\begin{subfigure}{0.5\textwidth}
        \resizebox{0.9\linewidth}{!}{
            \input{assets/img/diagramme/ZeitAnspassungEndlessA.latex}
        }
		\caption{Schleife}
    \end{subfigure}
    \begin{subfigure}{\textwidth}
        \resizebox{\linewidth}{!}{
            \input{assets/img/diagramme/ZeitAnspassungEndlessC.latex}
        }
		\caption{Überprüfung Urlaubstage}
	\end{subfigure}
	\caption{Anapssung der Gesamtzeit einer endlosen Kategorie}
	\label{fig:anpassung-endlos}
\end{figure}



\subsection{Zeitberechnungen}
Berechnungen mit Moment

\subsection{Darstellung von Zeiten}
Die Darstellung der Zeiten stellt eine designerische und eine technische Herausforderung.

\subsubsection{Design}
Die Zeiten sollte so dargestellt werden,
dass der Nutzer sie eindeutig verstehen kann und alle notwendigen Informationen erhält.
Trennt mann beispielsweise die Zeiteinheiten immer mit einem Doppelpunkt und zeigt nicht immer die Sekunden an,
so kann es zu Verwirrungen zwischen \enquote{12:23} als \enquote{12 Stunden 23 Minuten}
und \enquote{12:23} als \enquote{12 Minuten 23 Sekunden} kommen.

Um diese Ziele zu erreichen erfolgt die Darstellung nach den in \autoref{fig:darstellung-zeiten} beschriebenen Regeln.

\begin{figure}[ht!]
    \centering
    \resizebox{\textwidth}{!}{
        \input{assets/img/diagramme/ZeitDarstellung.latex}
    }
    \caption{Darstellung von Zeiten}
    \label{fig:darstellung-zeiten}
\end{figure}


% \begin{algorithm}
%     \caption{Darstellung von Zeiten}
%     \label{alg:darstellung-zeiten}
%     \uIf{Zeit $\geq$ 10 Tage}{
%         Darstellung: \enquote{$DD$d} \tcc*[r]{13d}
%     }
%     \uElseIf{Zeit $>$ 1.5 Tage}{
%         Darstellung: \enquote{$DD$d $HH$h} \tcc*[r]{04d 12h}
%     }
%     \Else{
%         \uIf{Doppelpunkt als Trenner}{
%             \If{Zeit $\geq$ 1 Stunde}{
%                 Darstellung: \enquote{$HH$:$MM$:$SS$} \tcc*[r]{02:12:24}
%             }
%             \Else{
%                 Darstellung: \enquote{$MM$:$SS$} \tcc*[r]{12:24}
%             }
%         }
%         \Else{
%             \If{Zeit $\geq$ 10 Stunden}{
%                 Darstellung: \enquote{$HH$h $MM$m} \tcc*[r]{12h 13m}
%             }
%             \uElseIf{Zeit $\geq$ 1 Stunde}{
%                 Darstellung: \enquote{$HH$h $MM$m $SS$s} \tcc*[r]{09h 24m 42s}
%             }
%             \Else{
%                 Darstellung: \enquote{$MM$m $SS$s} \tcc*[r]{13m 37s}
%             }
%         }
%     }

% \end{algorithm}

Hierdurch wird erreicht, dass bei Darstellung mit Doppelpunkt immer die Sekunden angezeit werden.
Bei Zeiten über anderhalbt Tagen wird die Zeit überhaupt nicht mit Doppelpunkten dargestellt.
In der alternativen Darstellung mit Buchstaben wird durch den Buchstaben die Zeiteinheit eindeutig zugeordnet.
Daher kann hier bei entsprechend hohen Werten auf die Darstellung der kleineren Einheiten verzichtet werden.

\subsubsection{Technisch}
Läuft gerade eine Zeitmessung so soll der aktuelle Stand auch dem Nutzer präsentiert werden.
Hierbei soll sich der Wert logischerweise regelmäßig aktualisieren,
was technisch bedeutet, dass die React-Komponente neu gezeichnet werden muss.
Solch eine \enquote{Neu-Zeichnung} wird in React durch eine Änderung des States der Komponente erreicht.
Durch Redux hat allerdings eigentlich keines der Komponenten einen State.
Statdessen gibt es einen State für die gesamte Anwendung.

In \cite{Timersin85:online} werden drei Alternativen vorgestellt um dieses Problem zu lösen:
\begin{enumerate}
    \item Die Komponente erhält einen internen State,
    \item Es werden Redux-Actions für die Komponente angelegt und bei Bedarf gestartet oder gestoppt.
    \item Ein Globaler Timer läuft im Hintergrund und die Komponente wird bei diesem Timer registiert.
\end{enumerate}

Da es sich in diesem Beispiel um eine reine Darstellung handelt,
wird Variante 1 implementiert, indem das Beispiel aus \cite{Timersin85:online} auf den Anwendungsfall angepasst wird.
\autoref{lst:timer} zeigt den Quellcode der Klasse \texttt{Timer} mit folgenden Aspekten:
\begin{description}
    \item[Props] Dies sind die Eigenschaften, welche vom Elternelement Übergeben werden.
    \texttt{startTime} ist die Zeit, ab welcher der Timer läuft.
    Wird auch die optionale \texttt{baseTime} gesetzt, so werden diese beiden verrechnet.
    Zudem können an die Komponente alle Props übergeben werden, welche auch an eine Textkomponente üebrgeben werden.
    Diese werden an die interne Textkomponente weitergereicht.
    Die Props sind nicht veränderlich.
    \item[State] Der State ist der veränderliche Teil der Komponente.
    Er beinhaltet die \texttt{timerId} um den Timer wieder stoppen zu können,
    sowie den darzustellenden Text.
    \item[Ablauf]
    Um den Text zu aktualisieren wird durchläuft die Komponente folgende Phasen:
    \begin{enumerate}
        \item Beim Erzeugen des Timers wird State initialisiert.
        \item Sobald die Komponente Teil der Darzustellenden Komponenten wurde (\texttt{componentDidMount}),
        wird ein Intervall gestartet, welcher sekündlich die \texttt{tick}-Methode aufruft.
        \item In der \texttt{tick} wird der neu darzustellende Text ermittelt und der State aktualisiert.
        \item Wird der Timer entfernt, so wird das Intervall gestoppt.
    \end{enumerate}
\end{description}

\lstinputlisting[
    caption={Timer.tsx},
    label=lst:timer,
    language=TypeScript,
    firstnumber=5,
    firstline=5
]{Timer.tsx}


\subsection{Dialoge, DatePicker}
Losgelöst von Redux-State

\subsection{Benachrichtigungen}\label{sec:besonderheiten-benachrichtigung}
Zur Darstellung der lokalen Benachrichtigungen wird das \emph{React Native Firebase's Notifications}-Modul genutzt.
Es bietet einfache Schnittstellen um Benachrichtigungen zu erzeugen,
sie direkt anzuzeigen oder für einen bestimmten Zeitpunkt zu planen.
Durch das Singleton \texttt{NavigationService} werden diese Schnittstellen weiter vereinfacht und auf den Anwendungsfall angepasst.
Die Klasse hat hierbei zwei Aufgaben: Das Senden bzw. Planen von Benachrichtigungen und die Behandlung eingehender Benachrichtigungen:

\subsubsection{Senden und Planen von Benachrichtigungen}
Hierfür werden verschiedene Funktionen zur Verfügung gestellt:
\begin{description}
    \item[buildNotification] erzeugt eine Benachrichtigung mit ID, Titel, Inhalt und Kanal\footnote{Ein Kanal wird von Android genutzt um Benachrichtigungen zu klassifizieren. In dieser Anwenung wird lediglich ein Default-Kanal verwendet.}.
    \item[scheduleNotification] erhält zusätzlich zu den obigen Informationen noch einen Zeitpunkt an welchem die Benachrichtigung erscheinen soll.
    \item[cancelNotificationIfExists] wird genutzt, um eine bereits geplante Benachrichtigung abzubrechen. Hierfür erhält es die ID der Benachrichtigung.
\end{description}

\subsubsection{Behandlung eingehender Benachrichtigungen}
Bei der Behandlung eingehender Benachrichtigungen gilt es,
verschiedene Situationen zu Betrachten und jeweils entsprechend zu reagieren \cite{Introduc87:online}.
\begin{description}
    \item[App wird durch eine Benachrichtigung gestartet]
    Läuft die App nicht, so werden bereits geplante Benachrichtigungen durch das Betriebssystem angezeigt.
    Klickt der Nutzer auf diese, so wird dadurch die App gestartet.
    In der App wird dafür gesorgt, dass die Benachrichtigung entfernt wird.

    \item[App wird durch eine Benachrichtigung geöffnet]
    Befindet sich die App im Hintergrund,
    dann wird eine Benachrichtigung ebenfalls durch das Betriebssystem angezeigt.
    Auch hier wird die Benachrichtigung durch die App entfernt.

    \item[Eine Benachrichtigung wird bei aktiver App empfangen]
    Ist die App aktiv, so wird eine Benachrichtigung zwar empfangen,
    aber nicht automatisch durch das Betriebssystem angezeigt.
    In diesem Fall geschieht dies durch Programmcode.

    \item[Benachrichtigung wird bei aktiver App angezeigt]
    In diesem Fall wird bei aktiver Applikation eine Benachrichtigung angezeigt.
    Bei lokalen Benachrichtigungen ist dies nur der Fall,
    wenn die Applikation \enquote{bewusst} eine Benachrichtigung anzeigt.
\end{description}

\noindent
Für jedes dieser Situationen wird im \texttt{NotificationService} ein sogenannte \emph{Listener} zur Verfügung gestellt.
In der Hauptkomponente \texttt{App.tsx} werden diese beim Start registiert und bei Beendigung abgemeldet.

\subsubsection{Tatsächliche Benachrichtigungen}
Wie auch schon bei der Zeiterfassung ist bei den Benachrichtigungen zwischen endloser und einer Intervall-Kategorien zu unterscheiden.
Bei einer endlosen Zeiterfassung wird angenommen,
dass man sich am Tagesbeginn im \enquote{Unterstunden-Bereich} befindet.
Startet der Nutzer Zeitmessung wird ermittelt wie lange diese laufen muss,
bis die Unterstunden \enquote{abgearbeitet} sind.
Für diesen Zeitpunkt wird eine entsprechende Benachrichtigung geplant.

Im Fall der Intervall-Kategorie geschieht dies analog.
Allerding berechnet wie lange die Messung laufen müsste,
um ausgehend vom akutellen Stand das Ziel zu erreichen.

In beiden Fällen wird die geplante Benachrichtigung abgebrochen,
sobald der Nutzer die Zeiterfassung pausiert.

Zusätzlich zu den implementierten Benachrichtigungen sind noch weitere denkbar.
Beispielsweise könnte in regelmäßigien Abständen an Pausen erinnert werden
oder nicht nur beim Erreichen von \enquote{0 Unterstunden},
sondern auch beim Erreichen der Sollarbeitszeit für diesen Tag eine Benachrichtigung erfolgen.
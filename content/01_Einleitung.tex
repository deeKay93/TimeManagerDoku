\section{Einleitung (Lea)}\label{sec:einleitung}
\subsection{Problemstellung}
\todo[inline]{Problemstellung: Was für ein Problem möchten sie wie lösen, wie haben andere das Problem gelöst}
\begin{itemize}
    \item Als Dualer Student reduzierte Arbeitszeit (z.B. 80\%)
    \item Zeitenmanagement ist sehr wichtig --> Arbeit und Uni
    \item Überblick über Über-/Unterstunden bei der Arbeit
    \item Ziele für das Lernen bei der Uni
\end{itemize}

\subsection{Anforderungen}
\todo[inline]{Anforderungen: Mindestens eine numerierte Liste}
\begin{itemize}
    \item Benutzerverwaltung
    \begin{itemize}
        \item Anmelde/Registrierungsmöglichkeiten (anonym, email/pw)
        \item Benutzerdaten
    \end{itemize}
    \item Konfiguration verschiedener Zeiterfassungskategorien
    \begin{itemize}
        \item Arbeitszeit pro Tag/Arbeitszeit pro Woche
        \item Erfassungsfreie Tage (Urlaub, Feiertage)
    \end{itemize}
    \item Erfassen der Zeit
    \begin{itemize}
        \item Kategorieauswahl
        \item Zeit starten und stoppen
        \item manuelle Zeiterfassung
        \item Berechnung des aktuellen Status (Total)
    \end{itemize}
    \item Darstellen der Ergebnisse
    \begin{itemize}
        \item Übersicht (z.B. der Tage)
        \item Detailansicht (übersicht über alle Zeitaufzeichnungen)
        \item Push-Benachrichtungen (Ziel erreicht)
    \end{itemize}
    \item Datenspeicherung mit Firebase
\end{itemize}

\todo[inline]{Auswahl React (bzw. native), redux, redux-saga, weil es auch auf Arbeit genutzt werden wird und man so Grundlagen bekommt}

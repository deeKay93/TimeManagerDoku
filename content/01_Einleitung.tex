\section{Einleitung}\label{sec:einleitung}
\subsection{Problemstellung}
Während eines Dualen Masterstudiums ist das Zeitenmanagement sehr wichtig. Häufig wird die
Arbeitszeit auf 80\% reduziert, um Zeit für Vorlesungen, Seminararbeiten und Klausuren zu schaffen.
Dabei kann es allerdings vorkommen, dass in manchen Wochen Überstunden aufgebaut werden, die in beispielsweise
Klausurwochen wieder abgebaut werden können. Die App soll es ermöglichen einen Überblick über die
Über- bzw. Minusstunden zu behalten.\\
Die bisherige Alternative zur Lösung des Problems sind Excel-Tabellen.
Diese haben allerdings den Nachteil, das der manuelle Aufwand für die Pflege der Einträge relativ hoch ist.
Dies liegt daran, dass die meisten verfügbaren Excel-Tabellen entweder für ein Studium oder für
das Zeitenmanagement von Arbeitnehmern angepasst sind. Durch zu viele und nicht notwendge Funktionalitäten,
wird das Zeitenmanagement nicht nur aufwändig sondern auch unnötig komplex.\\
Dieser Prototyp soll das Problem so lösen, dass der regelmäßige manuelle Aufwand für die Zeiterfassung
möglichst gering ist. Zusätzlich soll die App die Möglichkeit bieten,
durch setzen von Zeitzielen zum Lernen zu motivieren.

\subsection{Anforderungen}
\paragraph{Benutzerverwaltung:}
Mit der App muss es möglich sein, einen Benutzer neu zu registrieren oder mit einem bereits
vorhandenen Account anzumelden.
Die dafür notwendigen Daten müssen auf der Firebase Datenbank gespeichert werden.
Es soll zusätzlich möglich sein, den Benutzernamen auch nach der Registrierung zu verändern.
\paragraph{Konfiguration verschiedener Zeiterfassungskategorien}
Zeiten, die erfasst werden, sollen in benutzerspezifischen Kategorien organisiert sein.
Es soll zwei Kategorietypen geben:
\begin{itemize}
    \item \textbf{Endlose Zeiterfassung (Arbeitszeit-Szenario):} Hierfür werden
    Informationen wie Arbeitsstunden pro Woche und Arbeitstage benötigt.
    Das \enquote{Stundenkonto} soll auf die geforderte Arbeitszeit angepasst werden.
    \item \textbf{Intervall-Zeiterfassung (Lernzeit-Szenario):} Hierbei geht es darum, eine
    Zielstundenzahl anzugeben, die in einer bestimmten Zeit geleistet werden soll. Nach Ablauf dieses Intervalls
    wird die Gesamtzeit wieder zurückgesetzt und die Zeiterfassung beginnt von vorne.
\end{itemize}
\paragraph{Erfassungsfreie Tage:}
Neben den bisher genannten Eigenschaften, sollen auch erfassungsfreie Tage gepflegt werden können. Diese sollen
manuell eintragbar sein (zum Beispiel für Urlaubstage). Zusätzlich soll es die Möglichkeit geben,
Feiertage zu importieren, um den Aufwand für den Benutzer möglichst gering zu halten.
\paragraph{Zeiterfassung:}
Die Zeiterfassung, muss einer Kategorie zugeordnet sein. Der Benutzer soll die Erfassung starten und stoppen können.
Zusätzlich soll es auch nachträglich möglich sein, weitere Zeiten manuell hinzuzufügen.
Der Benutzer erhält als Ergebnis bei endloser Zeiterfassung das Stundenkonto und bei Intervall-Zeiterfassung die in diesem Intervall geleistete Zeit.
\paragraph{Darstellung der Ergebnisse:}
Die Ergebnisse der Zeiterfassung sollen für den Benutzer in einer Übersicht und in einer Detailansicht verfügbar sein.
Zusätzlich soll der Benutzer über Push-Benachrichtungen über das Erreichen eines Zieles imformiert werden.
\paragraph{Datenspeicherung:}
Die Daten sollen in einer Firebase-Datenbank gespeichert werden.

\section{Werkzeuge und Technologien (Lea)}
\paragraph{React-Native}

\todo[inline]{Unterschied React, React web React Native}
\todo[inline]{Diagramm Gesamtablauf}
\todo[inline]{- React}
\todo[inline]{- Redux}
\todo[inline]{- Redux-Saga und Firebase}


\paragraph{Typescript}
\todo[inline]{Typescript}
\subsection{Verwendete Bibliotheken}
Im folgenden Abschnitt, werden die in der Applikation verwendetetn Bibliotheken kurz vorgstellt.


\paragraph{react}
\todo[inline]{react}
Komponenten
\paragraph{react-native}
eigentlicher Renderer

\paragraph{react-native-firebase}
\todo[inline]{react-native-firebase}
Firebaseanbindung

\paragraph{react-native-material-dialog}
\todo[inline]{react-native-material-dialog}
Dialog im Material Style

\paragraph{react-navigation}
\todo[inline]{react-navigation}
Navigation im UI
%https://reactnavigation.org/
\paragraph{redux}
\todo[inline]{redux}
state Management

\paragraph{react-redux}

\todo[inline]{react-redux}
Verknüpfung von React und Redux

\paragraph{redux-form}
\todo[inline]{redux-form}
einfache Formulare
Die Bibliothek \textit{redux-form} unterstützt beim Verwalten der Redux-States in Formularen.

\paragraph{redux-saga}
\todo[inline]{redux-saga}
Asynchrone Aktionen

\paragraph{moment}
\textit{Moment.js} ist eine JavaScript-Bibliothek, die das Arbeiten mit Daten vereinfacht. Die Bibliothek ermöglicht es,
ein Datum zu formatieren, validieren und zu manipulieren. Dies ist besonders hilfreich bei der Auswertung und Darstellung der
Zeiterfassung. Das Datum muss hierzu in ein \textit{Moment} umgewandelt werden. \autoref{lst:moment-example} zeigt in einem kleinen Beispiel,
wie die Bibliothek in der Applikation verwendet wurde.
% \lstinputlisting[
%     caption=Beispiel für Moment.js,
%     label=lst:moment-example,
%     language=Typescript,
%     float=ht
% ]{moment.ts}

\paragraph{moment-duration-format}
Dies ist ein Plugin zusätzlich zu \textit{moment.js}. Dieses Plugin in notwendig, da die Dauer ein
grunsätzlich anderes Format hat als ein Datum. Es ermöglicht beispielsweise die Umwandlung einer Anzahl Stunden in Minuten.
In der Applikation wird dieses Format zur Darstellung der Zeiterfassungsergebnisse verwendet.



\paragraph{native-base}
\textit{NativeBase} ist ein Framework, das eine Schicht über React-Native darstellt. Das Framework bietet,
Komponenten für React-Native, die plattformspezifische Designs umsetzen. Diese Komponenten werden für die meisten UI-Komponenten der App
verwendet. %https://nativebase.io/
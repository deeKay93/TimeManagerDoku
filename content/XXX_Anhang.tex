\newpage
\appendix
\section{Datenbankstruktur}\label{app-datenbank}
    \begin{forest}
        pic dir tree,
        % where level=0{}{% folder icons by default; override using file for file icons
        %   directory,
        % },
        [/
            [users,directory
                [user1,file
                    [name]
                    [categories,directory
                        [categorie11,file
                            [name]
                            [total]
                            [target]
                            [...]
                            [times,directory
                                [2018-10-02T06:50:00.000Z,file
                                    [minutes]
                                    [started]
                                    [stopped]
                                ]
                                [...]
                            ]
                        ]
                        [...]
                    ]
                    [holidays,directory
                        [holiday1,file
                            [name]
                            [startDay]
                            [endDay]
                            [isFullDay]
                            [hours]
                        ]
                        [...]
                    ]
                ]
            ]
        ]
    \end{forest}

\section{LoginDeciderComponent}\label{app-LoginDecider}
\lstinputlisting[
    nolol,
    language=TypeScript
]{LoginDecider.tsx}

\newpage
\section{Reflexion}\label{app-reflexion}

\subsection{Reflexion Lea Poletin}

Durch die Entwicklung der Applikation konnte ich sehr viel über die Technologien React, React-Native, Redux und Redux-Saga lernen.
Bisher hatte ich lediglich die Kenntnisse aus dem Vorlesungsteil \textit{React-Native}. Mit Redux und Redux-Saga hatte ich
bisher keinen Kontakt und musste so zunächst die Konzepte erlernen.

Dabei profitierte sich besonders die Zusammenarbeit mit Daniel, der bereits theoretische Erfahrung mit diesen Themen besaß.
Anschließend war es bei der praktischen Umsetzung die größte Herausforderung, die verschieden Technologie-Konzepte
nicht zu vermischen.

Die Arbeit im Team war an sich sehr angenehm und hilfreich, denn sie  sorgte für Denkanstöße, Unterstützung bei Problemen und gegenseitige Motivation.
Die Herausforderungen der Aufgabenverteilung wurden so deutlich kompensiert.
Auch die Diskussionen zu Design-, Struktur- und Technologieentscheidungen, kosteten zwar Zeit, allerdings konnten so verschiedene Aspekte diskutiert und
Vor- und Nachteile abgewogen werden.

Die Entscheidung JavaScript mit Typescript zu ergänzen, war zunächst relativ aufwändig. Auch hier musste ich die verschiedenen Möglichkeiten zunächst kennenlernen,
 da ich Typescript bisher nicht verwendet habe. Es half allerdings die Programmiersprache Java zu kennen, da diese auch stark typisiert ist.
Die Typisierungen zahlten sich relativ schnell bei der Entwicklung aus, da so beispielsweise die richtigen Eigenschaften eines Objekts angezeigt wurden.

Bei der Entwicklung erschwerten besonders der Packager und der Debugger die Produktivität.
Diese hatten oftmals Fehler, die nur durch einen Neustart gelöst wurden. Auch der Emulator, den ich zu Beginn der Entwicklung
verwendete, zeigte einige Tücken, weshalb ich relativ schnell auf Hardware, meinem eigenen Smartphone, umgestiegen bin.

Die sehr modulare Ordnerstruktur hat den Vorteil, dass die einzelnen Einheiten klar voneinander abgegrenzt sind.
Jedoch bedeutet dies auch, dass bei der Entwicklung einer neuen Funktionalität, an vielen kleinen Stellen etwas verändert oder
hinzugefügt werden musste. Dies war besonders am Anfang sehr viel Aufwand, da hier die Applikation noch relativ klein war.

Die App, die als Ergebnis entstanden ist, ist für mich sehr zufriedenstellend. Wie im Fazit beschrieben, können allerdings
noch weitere Funktionalitäten ergänzt werden.

Insgesamt habe ich sehr viel über die verschiedenen Technologien gelernt, dass mir in meiner zu bei meiner zukünftigen Arbeit hilft.
Dort sollen einige dieser Konzepte bald eingesetzt werden.

\subsection{Reflexion Daniel Knecht}
Vor diesem Projekt hatte ich nur grundlegende Erfahrungen in TypeScript.
React Native und die Entwicklung  für mobile Geräte an sich habe ich erst in der Vorlesung kennen gelernt.
Mit React (Web), Redux und Redux Saga hatte ich bereits Berührungspunkte,
aber keine der Technologien praktisch angewandt.

Um die Technologien auf einfache Weise miteinander zu kombinieren haben wir zunächst eine
Beispielapplikation erstellt und diese nach und nach mit den Technologien angereichert.
So konnte eine gute technische Grundlage für die Anwendung geschaffen werden.

Für die eigentliche App wurden Schritt für Schritt neue Funktionalitäten hinzugefügt
und sich dabei das notwendige Wissen angeeignet.
Die Dokumentationen der Bibliotheken waren hier die zuverlässigste Quelle.

Während der Entwicklung waren drei Herausforderungen allgegenwärtig:

\textbf{Die verschiedenen Werkzeuge:}
So musste man beispielsweise den React Native Service mehrfach starten,
da die App nicht auf dem Gerät installiert werden konnte.
Außerdem konnten teilweise die Firebase-Daten nicht geladen werden,
wenn der Debugger aktiv war.

\textbf{Die schnelle Änderung der Pakete:}
React Native hat ein sehr aktives Ökosystem.
Das bedeutet leider auch,
dass sich viele Pakete sehr schnell ändern können.
Beispielsweise funktionierten die geplanten Benachrichtigungen in react-native-firebase Version 5.0.0-rc4 (Veröffentlicht am 20.09.2018) nicht korrekt.
In Version 5.0.0 (vom 26.09.2018) sind diese Probleme behoben. In diesem Beispiel wurden innerhalb etwa einer Woche 6 verschiedene Versionen veröffentlicht.

\textbf{Das UI-Design in React Native:}
Das Design in React Native hat viele Ähnlichkeiten zum bekannten CSS.
Aber es sind die Unterschiede, die einem zu schaffen machen.
In der Webentwicklung kann man Dank Chrome Developer Tools das CSS \enquote{live} manipulieren.
Eine ähnliche Funktionalität mit React-Native haben wir während der Entwicklung nicht gefunden.

Ungehindert dieser Herausforderungen war die Entwicklung gefüllt von kleinen Erfolgen.
Ein UI-Element, das reagiert wie erwartet,
eine Bibliothek, die sich nahtlos in das Projekt einbinden lässt,
jede kleine Verbesserung, jeder Schritt in Richtung vollwertiger Applikation motiviert.

Das Arbeiten im Team hat so seine Tücken.
Man muss sich abstimmen, die Aufgaben verteilen und Kompromisse für unterschiedliche Meinungen finden.
Ich würde das Team dennoch der Einzelarbeit vorziehen, denn man motiviert sich auch gegenseitig,
gleicht des anderen Schwächen aus und kann sich allgemein gegenseitig unterstützen.

Insgesamt bin ich mit dem Ergebnis der Arbeit sehr zufrieden.
Es gibt zwar noch einiges Verbesserungspotential aber die grundlegenden Funktionen sind vorhanden.
Alles Weitere ist \enquote{reine Schreibarbeit}.

Ich persönlich konnte das Projekt vor allem nutzen, um die verschiedenen modernen Technologien praxisnah einzusetzen.
So habe ich die Technologien kennen und anzuwenden gelernt.
Für meine zukünftige Arbeit im Unternehmen bedeutet dies,
dass ich mich heute schon mit Technologien intensiv auseinandergesetzt habe,
die andere Kolleginnen und Kollegen erst noch lernen müssen.
\newpage
\appendix
\section{Datenbankstruktur}\label{app-datenbank}
    \begin{forest}
        pic dir tree,
        % where level=0{}{% folder icons by default; override using file for file icons
        %   directory,
        % },
        [/
            [users,directory
                [user1,file
                    [name]
                    [categories,directory
                        [categorie11,file
                            [name]
                            [total]
                            [target]
                            [...]
                            [times,directory
                                [2018-10-02T06:50:00.000Z,file
                                    [minutes]
                                    [started]
                                    [stopped]
                                ]
                                [...]
                            ]
                        ]
                        [...]
                    ]
                    [holidays,directory
                        [holiday1,file
                            [name]
                            [startDay]
                            [endDay]
                            [isFullDay]
                            [hours]
                        ]
                        [...]
                    ]
                ]
            ]
        ]
    \end{forest}

\section{LoginDeciderComponent}\label{app-LoginDecider}
\lstinputlisting[
    nolol,
    language=TypeScript
]{LoginDecider.tsx}

\section{Reflexion}\label{app-reflexion}
Ausgangssituation: Entwicklung mit JS (kein Typescript). Bezüglich React Native ausschließlich die Vorkenntnisse aus der Vorlesung.

Ziel der App: React(-native)

\subsection{Reflexion Lea Poletin}
Durch die Entwicklung der Applikation konnte ich sehr viel über die verschiedenen Technologien lernen. 
Ich hatte bisher keine Erfahrung mit React, Redux, Redux-Saga und TypeScript. 
Dies hat den Einstieg für mich allerdings sehr schwer gestaltet, da die Konzepte nicht 
zwar kombiniert aber nicht vermischt werden sollten.\\
Zusätzlich erschwerte der Debugger und der Packager die Entwicklung. 
Diese liefen leider nicht zuverlässig und manchmal konnten Fehler so nicht 
gedebuggt werden. \\
Insgesamt war die Entwicklung der Applikation eine gute Erfahrung bei der ich viel gelernt habe. 
Besonders die Erfahrung mit React und Redux hilft mir bei meiner zukünftigen Arbeit, denn dort sollen
diese Konzepte bald eingesetzt werden.\\
Die App kann nach dieser Seminararbeit weiterentwickelt und produktiv eingesetzt werden. 
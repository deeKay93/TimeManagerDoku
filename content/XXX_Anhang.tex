\newpage
\appendix
\section{Datenbankstruktur}\label{app-datenbank}
    \begin{forest}
        pic dir tree,
        % where level=0{}{% folder icons by default; override using file for file icons
        %   directory,
        % },
        [/
            [users,directory
                [user1,file
                    [name]
                    [categories,directory
                        [categorie11,file
                            [name]
                            [total]
                            [target]
                            [...]
                            [times,directory
                                [2018-10-02T06:50:00.000Z,file
                                    [minutes]
                                    [started]
                                    [stopped]
                                ]
                                [...]
                            ]
                        ]
                        [...]
                    ]
                    [holidays,directory
                        [holiday1,file
                            [name]
                            [startDay]
                            [endDay]
                            [isFullDay]
                            [hours]
                        ]
                        [...]
                    ]
                ]
            ]
        ]
    \end{forest}

\section{LoginDeciderComponent}\label{app-LoginDecider}
\lstinputlisting[
    nolol,
    language=TypeScript
]{LoginDecider.tsx}

\section{Reflexion}\label{app-reflexion}
Ausgangssituation: Entwicklung mit JS (kein Typescript). Bezüglich React Native ausschließlich die Vorkenntnisse aus der Vorlesung.

Ziel der App: React(-native)

\subsection{Reflexion Lea Poletin}
Durch die Entwicklung der Applikation konnte ich sehr viel über die verschiedenen Technologien lernen. 
Ich hatte bisher keine Erfahrung mit React, Redux, Redux-Saga und TypeScript. 
Dies hat den Einstieg für mich allerdings sehr schwer gestaltet, da die Konzepte nicht 
zwar kombiniert aber nicht vermischt werden sollten.\\
Zusätzlich erschwerte der Debugger und der Packager die Entwicklung. 
Diese liefen leider nicht zuverlässig und manchmal konnten Fehler so nicht 
gedebuggt werden. \\
Insgesamt war die Entwicklung der Applikation eine gute Erfahrung bei der ich viel gelernt habe. 
Besonders die Erfahrung mit React und Redux hilft mir bei meiner zukünftigen Arbeit, denn dort sollen
diese Konzepte bald eingesetzt werden.\\
Die App kann nach dieser Seminararbeit weiterentwickelt und produktiv eingesetzt werden. 


\subsection{Reflexion Daniel Knecht}
Vor diesem Projekt hatte ich nur grundlegende Erfahrungen in TypeScript.
React Native und die Entwiclkung für Mobile Geräte an sich habe ich erst in der Vorlesung kennen gelernt.
Mit React (Web), Redux und Redux Saga hatte ich bereits Berührungspunkte,
aber keine der Technologien praktisch angewandt.

Um die Technologien auf einfache Weise miteinander zu kombinieren haben wir zunächst eine
Beispielappliaktion erstellt und diese nach und nach mit den Technologien angereichert.
So konnte eine gute technische Grundlage für die Anwendung geschaffen werden.

Für die eigetliche App wurden Schritt für Schritt neue Funktionalitäten hinzugefügt
und sich dabei das notwendige Wissen angeeignet.
Die Documentationen der Bibiotheken waren hier die zuverlässigste Quelle.

Während der Entwicklung waren drei Herausforderungen allgegenwärting:
Die verschiedenen Wekrzeuge: So musste man beispielsweise den React Native Service mehrfach starten,
da die App nicht auf dem Gerät installiert werden konnte.
Außerdem konnten Teilweise die Firebase-Daten nicht geladen werden,
wenn der Debugger aktiv war.

Die schelle Änderung der Pakete:
React Native hat ein sehr aktives Ökosystem.
Das bedeutet leider auch,
dass sich viele Pakete sehr schnell ändern können.
Beispielsweise funktionierten die geplanten Benachrichtigungen in react-native-firebase Version 5.0.0-rc4 (Veröffentlicht am 20.09.2018) nicht korrekt.
In Version 5.0.0 (vom 26.09.2018) sind diese Probleme behoben. In diesem Beispiel wurden innerhalb etwa einer Woche 6 verschiedene Versionen veröffenlticht.

Das UI-Design in React Native:
Das Design in React Native hat viele Ähnlichkeiten zum bekannten CSS.
Aber es sind die Unterschiede, die einem zu schaffen machen.
In der Webentwicklung kann man Dank Chrome Developr Tools das CSS \enquote{live} manipulieren.
Eine ähnliche Funktionalität mit React-Native haben wir während der Entwicklung nicht gefunden.

Ungehindert dieser Herausforderungen war die Entwicklung gefüllt von kleinen Erfolgen.
Ein UI-Element das reagiert wie erwartet,
eine Bibliothek, die sich nahtlos in das Projekt einbinden lässt,
jede kleine Verbesserung, jeder Schritt in Richtung vollwertiger Applikation motiviert.

Das Arbeiten im Team hat so seine Tücken.
Man muss sich abstimmen, die Aufgaben verteilen und Komporomisse für unterschiedliche Meineungen finden.
Ich würde das Team dennoch der Einzelarbeit vorziehen, denn man motiviert sich auch gegenseitig,
gleicht des anderen Schwächen aus und kann sich allgemein gegenseitig unterstützen.

Insgesamt bin ich mit dem Ergebnis der Arbeit sehr zufrieden.
Es gibt zwar noch einiges Verbesserungspotential aber die grundlegenden Funktionen sind vorhanden.
Alles weitere ist \enquote{reine Schreibarbeit}.

Ich persönlich konnte das Projekt vor allem nutzen, um die verschiedenen modernen Technologien praxisnah einzusetzen.
So habe ich die Technologien kennen und anzuwenden gelernt.
Für meine zukünftige Arbeit im Unternehmen bedeuted dies,
dass ich mich heute schon mit Technologien intensiv auseinander gesetzt habe,
die andere Kolleginnen und Kollegen erst noch lernen müssen.
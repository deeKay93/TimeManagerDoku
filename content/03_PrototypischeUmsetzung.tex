\section{Prototypische Umsetzung}\label{sec:prototyp}

\subsection{Werkzeug- und Technologieauswahl (Lea)}
\todo[inline]{Werkzeug- und Technologieauswahl}
\begin{itemize}
    \item React-Native (eigentlicher Renderer)
    \item TypeScript
    \item redux 
    \item firebase und redux-saga
\end{itemize}

\todo[inline]{Auswahlen/Entscheidungen für den Prototyp}
\begin{itemize}
    \item Übersicht über Bibliotheken
    \begin{itemize}
        \item React-Native (eigentlicher Renderer)
        \item React (Componenten)
        \item react-native-material-dialog (Dialoh material like)
        \item react-native-firebase (Firebaseanbindung)
        \item react-navigation (Navigation im UI)
        \item react-redux (Verknüpfung von React und Redux)
        \item redux (State Management)
        \item redux-form (Einfache Formulare)
        \item redux-saga (Asynchrone Aktionen)
        \item moment (Zeitberechnung)
        \item moment-duration-format (Formatierung von Dauer und Datum)
        \item native-base (Ui-Componenten Material Like)
    \end{itemize}
    \item sehr viele modulare elemente
    \item firebase ersetzt durch rnfirebase
\end{itemize}

\subsection{Besonderheiten (Daniel)}
\todo[inline]{Erwähnenswerte Besonderheiten bei der Umsetzung}
\begin{itemize}
    \item Codestruktur UI
    \begin{itemize}
        \item screens
        \item components
        \item forms
        \item container
    \end{itemize}
    \item Codestruktur Redux
    \begin{itemize}
        \item reducer
        \item store
        \item action
    \end{itemize}
    \item Codestruktur Redux-Saga und Firebase
    \begin{itemize}
        \item sagas
        \item api
    \end{itemize}
    \item Zeiterfassung
    \item Notifications
    \item Dialogs
    \item Databasestructure
    \begin{itemize}
        \item Firebaserules
        \item Documents und Collections
    \end{itemize}
\end{itemize}
\todo[inline]{Ergebnis}
\section{Verzeichnisstruktur}
In \autoref{fig-verzeichnis} sind die ersten zwei Ebenen der Verzeichnisstruktur dargestellt.
Auf die Auflistung automatisch erstellter und nicht veränderter Dateien wird verzichtet.
Der Zweck der Einzelnen Ordner und Dateien wird im Folgenden beschrieben.

\begin{figure}[H]
    \centering
    \begin{forest}
        pic dir tree,
        where level=0{}{% folder icons by default; override using file for file icons
          directory,
        },
      [/
        [App.tsx,file]
        [global.d.ts,file]
        [index.js,file]
        [android]
        [ios]
        [src]
      ]
   \end{forest}
   \begin{forest}
    pic dir tree,
    where level=0{}{% folder icons by default; override using file for file icons
      directory,
    },
    [/src
        [store.ts,file]
        [actions]
        [api]
        [assets]
        [components]
        [container]
        [forms]
        [reducers]
        [sagas]
        [screens]
        [utils]
    ]
\end{forest}
    \caption{Verzeichnisstruktur}
    \label{fig-verzeichnis}
\end{figure}

\paragraph{Oberste Ebene}
\begin{description}
    \item[App.tsx]
    App.tsx ist die Zentrale Datei der Applikation. In ihr wird:
    \begin{itemize}
        \item die Navigation initialisiert,
        \item die Behandlung von Nachrichten eingerichtet,
        \item die Anmeldung des Nutzers sichergestellt,
        \item der Redux-Store mit der App verknüpft.
    \end{itemize}
    \item[global.d.ts]
    Diese Datei wird genutzt um TypeScript-Typen der gesamten Anwendung zur Verfügung zu stellen.
    Hierdurch wird die Arbeit mit komplexeren Datentypen erheblich erleichtert.
    \item[index.js]
    In der Datei wird die Anwendung \enquote{registiert}. Hierdurch werden die React-Native-Componenten mit der nativen Applikation verknüpft.
    Diese automatisch generierte Funktion wird ergänzt durch einen sogenannten \emph{Polyfill}, der genutzt wird,
    um fehlende JavaScript-Funktionalitäten zu ermöglichen. \cite{undefine14:online}
    \item[android]
    Dieser Ordner enthält den Quellcode für die native Android App.
    Wird nativer Code - beispielsweise von React-Native-Firebase - genutzt,
    so muss er hier hinzugefügt werden.
    \item[ios]
    Dieser Ordner ist das Pendant zu vorigem Ordner.
    Da sich dieser Programmentwurf auf Android-Geräte beschränkt,
    bleibt er von manuellen Eingriffen unberührt.
    \item[src]
    In diesem Ordner befindet sich der größte Teil des relevanten Quellcodes.
    Dessen Inhalt wird im folgenden näher betrachtet.
\end{description}

\paragraph{src-Ordner}
Die Aufteilung des \emph{src}-Ordners orientiert sich an dem verbreiteten \emph{Rails-style} \cite{CodeStru4:online}.
Dies bedeutet, dass für jede \enquote{Art} von Bausteinen der Applikation ein eigener Ordner angelegt wird.
Somit ergibt sich folgende Struktur:

\begin{description}
    \item[store.ts]
    In der einzigen Datei direkt im src-Ordner werden die gesamten Reducer zum Redux-Store zusammengefasst und mit
    Redux-Saga verknüpft.
    \item[actions]

    \item[api]
    \item[assets]
    \item[components]
    \item[container]
    \item[forms]
    \item[reducers]
    \item[sagas]
    \item[screens]
    \item[utils]
\end{description}


% \item Codestruktur UI
% \begin{itemize}
%     \item screens
%     \item components
%     \item forms
%     \item container
% \end{itemize}
% \item Codestruktur Redux
% \begin{itemize}
%     \item reducer
%     \item store
%     \item action
% \end{itemize}
% \item Codestruktur Redux-Saga und Firebase
% \begin{itemize}
%     \item sagas
%     \item api
% \end{itemize}

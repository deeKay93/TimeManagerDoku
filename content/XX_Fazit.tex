\section{Fazit}\label{sec:fazit}
Betrachtet man die eingangs gestellten Anforderungen,
so ist festzustellen, dass alle grundeleged erfüllt sind.
Ebenso ist jedoch anzumerken, dass es in jedem Aspekt potential zur Verbesserung und Erweiterung gibt.

\paragraph{Benutzerverwaltung}
Benutzer können sich per E-Mail und Passwort registrieren und anmelden oder einen anonymen Zugang nutzen.
Allerdings wird lediglich der anonyme Nutzer beim Abmelden gelöscht.
Ein registrierter Nutzer kann nach aktuellem Stand nicht wieder entfernt werden.

\paragraph{Konfiguration verschiedener Zeiterfassungskategorien}
Intervall- und endlose Kategorien, sowie erfassungsfreie Tage können in der Applikation angelegt werden.
Bei Kategorien ist zudem eine Anpassung der Einstellungen möglich.
Es gibt jedoch auch hier Einschränkungen: So können weder Kategorien, noch erfassungsfreie Tage gelöscht,
noch freie Tage in der Vergangenheit angelegt werden.

\paragraph{Zeiterfassung}
Der Anwender kann die Zeit einer Kategorie durch Starten und Stoppen der Zeitmessung und ebenfalls manuell erfassen.
Die manuelle Erfassung ist auch in der Vergangenheit möglich.
Problematisch hierbei ist, dass nicht validiert wird, ob sich die Zeiten einer Kategorie überschneiden.
Außerdem sollten insbesondere Randfälle wie das neue anlegen einer Kategorie oder die Messung über Intervallgrenzen hinaus intensiv gestestet werden.

\paragraph{Darstellung der Ergebnisse}
In der App wird jeweils eine tageweiße Übersicht und die genauen Messungen einer Kategorie dargestellt.
Wünschenswert sind hier weitere Darstellungungen.
Beispielsweise Diagramme, Informationen über Intervalle und verschiedene Granularitäten sind hier denkbar.
Ein weiteres Problem ist, dass die Erfaussungen immer komplett aus der Datenbank geladen werden.
Ein stufenweißes Laden der Daten nach Bedarf (\emph{Pagination}) wäre hier eine Lösung.

\paragraph{Datenspeicherung}
Wie gefordert werden die Daten online im Firestore von Firebase gespeichert.
Hierbei ist anzumerken, dass die Abrechnung dokumentenweiße erfolgt.
So ist die Anzahl der Lese-, Schreib- und Löschvorgänge im kostenlosen Plan limitiert.
Auch wenn diese Kontingente für die Entwicklung und den Gebrauch in kleinem Rahmen ausreichen,
sollte für produktive Applikationen untersucht werden,
ob sich die Vorgänge durch geschichte Datenstrukturierung optimieren lassen.

\paragraph{Ausblick}
Grundlegend ist die App funktionsfähig.
Damit sie auch produktiv eingesetzt werden kann gilt es zunächst,
die bereits beschriebenen fehldenden Funktionen zu implementieren.

Zusätzlich sollte die Stabilität durch manuelle,
aber auch automatisierte Tests sichergestellt werden.

Hat die Anwendung dann einen produktiven Status erreicht,
so sind weitere Funktionalitäten wie die Nutzung auf mehreren Geräten, eine Exportfunktion,
Unterkategorien oder die Unterstützung von iOS-Geräten Ddnkbar.
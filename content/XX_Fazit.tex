\section{Fazit (Daniel)}\label{sec:fazit}
Betrachtet man die eingangs gestellten Anforderungen,
so ist festzustellen, dass alle grundeleged erfüllt sind.
Ebenso ist jedoch anzumerken, dass es in jedem Aspekt potential zur Verbesserung und Erweiterung gibt.

\paragraph{Benutzerverwaltung}
Benutzer wird per Mail oder Anonym angelegt
Kann Seinen Namen Ändern
Problem: Mail-Nutzer nicht wieder löschbar

\paragraph{Konfiguration verschiedener Zeiterfassungskategorien}
Kategorien und Erfassungsfreit Tage können gepflegt (deutsche sogar importiert) werden.
Problem: Nicht wieder löschbar, Freie Tage nur in Zukunft (wegen Berechnungen)

\paragraph{Zeiterfassung}
Start /Stop und Manuell/nachträglich geht (auch in Vergangenheit).
Problem: Überschneidungen möglich. Intensives Testen um Randfälle zu betrachten (z.B. Zeiterfassung über INtervallgrenze hinaus)

\paragraph{Darstellung der Ergebnisse}
Nutzer erhält Tageweiße Übersicht und Details über jede erfassung.
Probleme: Aktuell alle daten geladen (keine pagination), keine Information über Intervalle,
Verschiedene Diagramme und GRanularitäten wünschenswert (z.B. Übersicht wann tägliche sollarbeitszeit erreicht wurde)

\paragraph{Datenspeicherung}
Daten werden Online gespeichert,
Firestore rechnet Dokumentenweiße ab -> Schreib- und Lesevorgänge optimieren
Problem (löschen Fehlt)


Ausblick:
Fehlende Funktionen implementieren
Tests implementieren
Nutzung auf mehreren Geräten
Weitere Funktionen:
    Export?
    Unterkategorien? (z.B. Arbeit -> Mails, Meetings, ...)
    Sortierung der Kategorien

\todo[inline]{erzielte Artefakte in Strichaufzählung}
